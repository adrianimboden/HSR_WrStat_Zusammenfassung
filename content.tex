%%%%%%%%%%%%%%%%%%%%%%%%%%%%%%%%%%%%%%%%%%%%%%%%%%%%%%%%%%%%%%%%%%%%%%%%
\section{Kombinatorik (Zählen)}

\subsection{Produktregel: Die Für–jedes–gibt–es–Regel (FJGE)}
Fur jede der $n_1$ Möglichkeiten gibt es eine von der
ersten Position unabhängige Anzahl $n_2$ Möglichkeiten für den Rest.
\[ = n_1 \cdot n_2 \text{ Möglichkeiten } \]

\subsection{Permutationen: Reihenfolge}
Auf wieviele Arten kann man $n$ Objekte anordnen? (Anzahl Anordnungen)
\[ = n! \text{ Arten} \]

\subsection{Kombinationen: Auswahl}
Auf wieviele Arten kann man $k$ Objekte aus $n$ auswählen?
\[ = C^n_k=\binom{n}{k} = \frac{n!}{k!(n-k)!} \text{ Arten} \]
Dieser "`Binomialkoeffizient"' lässt sich auf dem Taschenrechner
TI-36XII mit \texttt{n nCr k} (unter \texttt{PRB}), mit dem Voyage 200
mit \texttt{nCr(n,k)}, in Sage mit \texttt{binomial(n,k)} und in 
Octave/Matlab mit \texttt{nchoosek(n,k)} berechnen.

Auf wieviele Arten kann man $k$ Mal eine Auswahl aus $n$ Objekten
treffen?
\[ = n^k \text{ Arten} \]
